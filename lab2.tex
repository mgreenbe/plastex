\documentclass[12pt]{amsart}
\usepackage{amsmath, amsthm, amssymb}
\newcommand{\RR}{\mathbb{R}}
\newcommand{\ZZ}{\mathbb{Z}}
\DeclareMathOperator{\rref}{rref}
\DeclareMathOperator{\nullity}{nullity}
\DeclareMathOperator{\Span}{span}
\DeclareMathOperator{\rank}{rank}

\newtheorem{theorem}{Theorem}[section]
\newtheorem{thdf}{Theorem-Definition}[section]
\newtheorem{corollary}[theorem]{Corollary} \newtheorem{lemma}[theorem]{Lemma}
\theoremstyle{definition} \newtheorem{definition}[theorem]{Definition}
\newtheorem{provdef}[theorem]{Provisional definition}
\newtheorem{definitions}[theorem]{Definitions}
\newtheorem{remark}[theorem]{Remark} \newtheorem{remarks}[theorem]{Remarks}
\newtheorem{example}[theorem]{Example}
\newtheorem{exdef}[theorem]{Example-Definition}
\newtheorem{exercise}[theorem]{Exercise}

\newcommand{\ba}{\mathbf{a}}
\newcommand{\bb}{\mathbf{b}}
\newcommand{\bc}{\mathbf{c}}
\newcommand{\bd}{\mathbf{d}}
\newcommand{\be}{\mathbf{e}}
\newcommand{\bi}{\mathbf{i}}
\newcommand{\bu}{\mathbf{u}}
\newcommand{\bv}{\mathbf{v}}
\newcommand{\bw}{\mathbf{w}}
\newcommand{\bx}{\mathbf{x}}
\newcommand{\by}{\mathbf{y}}
\newcommand{\bz}{\mathbf{z}}
\newcommand{\bzero}{\mathbf{0}}

\newcommand{\bas}{\ba_1,\ldots,\ba_n}
\newcommand{\mat}[1]{\begin{bmatrix}#1\end{bmatrix}}
\newcommand{\Rmn}{\RR^{m\times n}} \newcommand{\Rmm}{\RR^{m\times m}}
\newcommand{\Rnn}{\RR^{n\times n}}
\newcommand{\spn}[1]{\Span\left(#1\right)}

\newcommand{\setstuff}{\setlength{\parskip}{0.5em}\setlength{\parindent}{0em}\setlength{\itemsep}{0.5em}}

\begin{document}
\title{MATH 311 -- Winter 2018 -- Lab 2}
\maketitle

\begin{enumerate}

  \setlength{\itemsep}{1em}

  \item Critique the following incorrect/nonsensical statements.

    \begin{enumerate}
      \setlength{\itemsep}{0.5em}
      \bigskip
      \item If $N(A)=\{0\}$ then $A$ is linearly independent.

      \item Vectors $\ba_1,\ldots,\ba_n$ are linearly dependent when all of their linear combinations are trivial.

      \item $\spn{\ba_1,\ldots,\ba_n}$ is a linear combination of $\ba_1,\ldots,\ba_n$.

      \item A vector $\bb$ belongs to the column space of $A$ if and only if $A\bx=\bb$ has a unique solution.

      \item $\bb$ is a column space of $A$ if and only if $A\bx=\bb$ has a solution.

      \item The columns of $A\in\RR^{m\times n}$ form a basis of $\RR^n$ if $N(A)$ is the zero subspace of $\RR^m$ and $C(A)$ spans $\RR^n$.
    \end{enumerate}

  \item Are the vectors
    \[
      \ba_1 = \mat{1\\-2\\-1\\3},\quad \ba_2=\mat{-3\\2\\1\\-5},\quad
      \ba_3 = \mat{4\\-2\\-1\\6}
    \]
    linearly independent? If not, write one of the $\ba_j$ as a linear combination of the other two.

  \item 
    \[
      A =\begin{bmatrix}-1 & 2 & 4 & 3\\2 & 0 & 4 & -2\\1 & 3 & 11 & 2\\0 & -1 & -3 & -1\end{bmatrix}.
    \]
    \bigskip
    \begin{enumerate}
      \setlength{\itemsep}{0.5em}
      \item Identify the pivot columns of $A$. Write the nonpivot columns of $A$ as as linear combinations of its pivot columns.
      \item Find a basis of $N(A)$.
    \end{enumerate}

  \item Let $\ba_1,\ba_2,\ba_3\in\RR^m$ and let 
    \[
    \bb_1=\ba_1,\quad \bb_2=\ba_1+\ba_2,\quad \bb_3=\ba_1+\ba_2+\ba_3.
    \]
    \begin{enumerate}
  \item Prove:
    \[
      \spn{\ba_1,\ba_2,\ba_3} = \spn{\bb_1,\bb_1,\bb_1}.
    \]
  \item Prove that $\ba_1$, $\ba_2$, and $\ba_3$ are linearly independent if and only if $\bb_1$, $\bb_2$, and $\bb_3$ are.
  \end{enumerate}


  \item Prove the statement or give a counterexample
    \bigskip
    \begin{enumerate}
      \setlength{\itemsep}{0.5em}
    \item $A$ and $\rref(A)$ have the same column space.

    \item $A$ and $\rref(A)$ have the same nullspace.
    \end{enumerate}
\end{enumerate}
\end{document}
